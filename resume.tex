\documentclass[multipage]{my_resume}
%\documentclass{my_resume}

\begin{document}

\pagenumbering{gobble}% Remove page numbers (and reset to 1)

\HeaderBar{%
	Jonathan Spitz, Ph.D.
}{%
	A creative roboticist, with a passion for solving complex problems.\\%
	Skilled researcher helping humanoid robots learn how to move.
}{%
	http://jonathanspitz.weebly.com/
}{%
	https://www.linkedin.com/in/spitzjonathan/
}{%
	+33-624-190544
}{%
	spitz.jonathan@gmail.com
}

\MainContent{white}{
	\section*{Professional experience}
	\PositionEntry{%
		February 2017 -- Present%
	}{%
		Postdoctoral researcher at Inria%
	}{%
		Nancy, France%
	}{%
		\begin{itemize}
			\item Developed whole-body control algorithms for humanoid robots in MATLAB/C++.%
			\item Implemented a machine learning approach to improve the controller's performance within a few trials.%
			\item Worked with a simulated and real iCub humanoid robot, possessing over 50 degrees of freedom.%
			\item Aided the development of a miniature exploration robot, as a secondary project.
		\end{itemize}
	}\EntrySpacing
	
	\PositionEntry{%
		June 2016 -- December 2016%
	}{%
		Algorithm engineer at Cogniteam%
	}{%
		Petach Tikva, Israel%
	}{%
		\begin{itemize}
			\item Worked on computer vision algorithms for autonomous navigation.%
			\item Designed a simulation environment to generate artificial data for algorithm testing.%
			\item Implemented structure-from-motion algorithms using stereo and multiple-view approaches.%
		\end{itemize}%
	}\EntrySpacing%

	\PositionEntry{%
		March 2015 -- October 2015%
	}{%
		Finalist at BizTEC Entrepreneurship Competition%
	}{%
		Haifa, Israel%
	}{%
		\begin{itemize}
			\item Participated in the summer-school and accelerator stages of the competition. One of 12 finalists out of over 80 teams.%
			\item Worked on all areas of the lean canvas to bring a robotic toy to market.%
			\item Robots received coverage at \href{http://techcrunch.com/2015/01/14/designer-builds-cute-little-3d-printed-robots-you-know-for-kids/}{TechCrunch}, \href{http://spectrum.ieee.org/automaton/robotics/robotics-hardware/video-friday-humanoids-sing-eyeball-robot-kuka-ping-pong}{IEEE Spectrum}.
		\end{itemize}%
	}\EntrySpacing%

	\PositionEntry{%
		October 2014 -- May 2016%
	}{%
		Rapid prototyping of walking robots%
	}{%
		Haifa, Israel%
	}{%
		\begin{itemize}
			\item Developed 3D-printed walking and wheeled robots including mechanical and electronic components.%
			\item Designed a compliant biped robot based on series-elastic actuators.%
			\item Programmed control loops and Bluetooth communication on
			Arduino.%
%			\item Robots received coverage at \href{http://techcrunch.com/2015/01/14/designer-builds-cute-little-3d-printed-robots-you-know-for-kids/}{TechCrunch}, \href{http://spectrum.ieee.org/automaton/robotics/robotics-hardware/video-friday-humanoids-sing-eyeball-robot-kuka-ping-pong}{IEEE Spectrum}.
		\end{itemize}%
	}\EntrySpacing%

	\PositionEntry{%
		August 2012 -- June 2013%
	}{%
		DARPA Virtual Robotics Challenge%
	}{%
		Haifa, Israel%
	}{%
		\begin{itemize}
			\item Created controllers for locomotion of the ATLAS humanoid robot.
			\item Worked in Ubuntu Linux using ROS-Gazebo and C++/python.
			\item Completed the dismounted mobility task (mud-pit, hills and debris
			areas) using robust crawling gaits.
			\item Created a robust, dynamic approach to enter the car using ATLAS.
		\end{itemize}
	}\EntrySpacing

	\PositionEntry{%
		April 2009 -- December 2009%
	}{%
		Undergraduate Project: Robotics Simulation Environment (ROSIE)%
	}{%
		Haifa, Israel%
	}{%
		\begin{itemize}
			\item Created a robotics simulation environment for the Robotics 101 course, coded in C++ and OpenGL.
			\item Features included forward and backward kinematics, path planning and Jacobian calculation. 
		\end{itemize}
	}\EntrySpacing

	\PositionEntry{%
		August 2007 -- September 2007%
	}{%
		Intern at Guzik Technical Enterprises%
	}{%
		Mountain View, CA, USA%
	}{%
		\begin{itemize}
			\item Designed SolidWorks models for a new product, including a rack, a motherboard and several connected
			boards.
			\item Worked with the electrical engineering department to select and position connectors for all the project's boards.
		\end{itemize}
	}\EntrySpacing

	\section*{Education}
	\DiplomaEntry{%
		October 2009 -- October 2015
	}{
		Ph.D. in Mechanical Engineering%
	}{%
		\textbf{,} Direct Ph.D. track for excellent students, GPA: 94/100.%
	}{
		Technion, Faculty of Mechanical Engineering%
	}{
		Haifa, Israel%
	}{
		Research Title: Bio-Inspired Controllers for Dynamic Locomotion\\%
		Advisor: Assoc. Prof. Miriam Zacksenhouse%
		\begin{itemize}
			\item Developed biologically inspired controllers that generate stable walking gaits for simple and complex biped models.
			\item Improved robustness ten-fold using minimal environmental feedback.
			\item Evolved a range of different gaits using multi-objective genetic	algorithms.
		\end{itemize}
	}\EntrySpacing
	
	\SimpleDiplomaEntry{%
		March 2005 -- February 2010%
	}{
		B.Sc. in Mechanical Engineering%
	}{%
		Technion, Faculty of Mechanical Engineering%
	}{
		Haifa, Israel%
	}\EntrySpacing
}{
	\section*{Other skills}
	\SkillEntry{Software}{MATLAB (+10 years), Arduino (+5 years), C++ (+5 years), Python (4 years), ROS (1 year).}
	\SkillEntry{Hardware}{SolidWorks (+10 years), 3D printing (2 years), Rapid prototyping (2 years).}
	\SkillEntry{Languages}{Spanish (native), English (near native), Hebrew (near native), German (basic proficiency), French (basic proficiency).}%
	\vspace{-8pt}
	
	\section*{Fellowships and awards}
	\underline{June 2014}: Late Prof. Roland Weill Award for excellence in doctoral research.\\%
	\underline{December 2012}: Fine's scholarship for 1\textsuperscript{st} year Ph.D. students.\vspace{0.15em}\\%
	\underline{March 2007}: Dean's Excellence Award for outstanding GPA (4\textsuperscript{th} semester).
	
	\section*{Patents}
	\BibEntry{1}{``Robot, device and a method for a central pattern generator (CPG) based control of a movement of the robot'', United States Patent Application 20140031986, January 2014.}
	
	\section*{List of publications}
	\underline{\smash{Refereed papers in Journals}}\vspace{0.15em}\\%
	\BibEntry{1}{J. Spitz, R. Yakar, M. Zacksenhouse,``Machine Learning Tools Facilitate Parameter Tuning of Networks of Matsuoka Neurons'', in preparation, August 2017.}\\%
	\BibEntry{2}{J. Spitz, A. Evstrachin, M. Zacksenhouse,``Minimal Feedback to a Rhythm Generator Improves the Robustness to Slope Variations of a Compass Biped'', Bioinspiration \& Biomimetics, August 2015.}\\%
	\BibEntry{3}{J. Spitz, E. Sidorov, M. Zacksenhouse, ``Humanoids Can Take Advantage of Crab-Walking Gaits'', International Journal of Humanoid Robotics, December 2014.}
	\EntrySpacing

	\noindent
	\underline{\smash{Refereed papers in Conference Proceedings}}\vspace{0.15em}\\%
	\BibEntry{1}{J. Spitz, K. Bouyarmane, S. Ivaldi, J.B. Mouret, ``Trial-and-Error Learning of Repulsors for Humanoid QP-based Whole-Body Control'', submitted to IEEE-RAS
		Humanoids. 2017.}\\%
	\BibEntry{2}{J. Spitz, Y. Or, M. Zacksenhouse, ``Towards a Biologically Inspired Open Loop Controller for Dynamic Biped Locomotion'', In Proceedings of the 2011 International Conference on Robotics and Biomimetics.}

	\section*{List of conferences}
	\underline{\smash{Oral Presentations}}\vspace{0.15em}\\%
	\BibEntry{1}{``Bio-Inspired Controllers for Dynamic Walking: CPG Enhanced with Minimal Feedback'', Israeli Conference on Robotics, 2013.}\\%
	\BibEntry{2}{``A Biologically Inspired Controller for Dynamic Bipedal Locomotion'', Graduate Students in Control Conference, 2012.}\\%
	\BibEntry{3}{``Towards a Biologically Inspired Open Loop Controller for Dynamic Biped Locomotion'', IEEE International Conference on Robotics and Biomimetics, Phuket, Thailand, 2011.}
	\EntrySpacing
	
	\noindent
	\underline{\smash{Poster Presentations}}\vspace{0.15em}\\%
	\BibEntry{1}{``Analytical and numerical tools for limit-cycle evaluation in hybrid dynamical system'', Dynamic Walking Conference, ETH Z\"urich, Switzerland, 2014.}\\%
	\BibEntry{2}{``Bio-Inspired Controllers for Dynamic Locomotion'', Dynamic Walking Conference, CMU Pittsburgh, USA, 2013.}\\%
	\BibEntry{3}{``A Biologically Inspired Controller For Dynamic Biped Locomotion'', The Eighth Computational Motor Control Workshop at Ben-Gurion University of the Negev, Israel, 2012.}
	
	\section*{Contact information for references}
	\begin{itemize}
		\item Jean-Baptiste Mouret, Ph.D. (Research Director at Inria), \EmailLink{jean-baptiste.mouret@inria.fr}
		\item Serena Ivaldi, Ph.D. (Research Scientist at Inria), \EmailLink{serena.ivaldi@inria.fr}
		\item Miriam Zacksenhouse, Ph.D. (Assoc. Prof. Technion), +972-54-5820404, \EmailLink{mermz@technion.ac.il}
%		\item Yizhar Or, Ph.D. (Assoc. Prof. at Technion), +972-77-8875493, 	
%		\EmailLink{izi@technion.ac.il}
		\item Ari Yakir (Project Manager at Cogniteam), \EmailLink{ari@cogniteam.com}
		\item Lauri Viitas (VP of Product and Business Development at Guzik), +1-650-625-8000, \EmailLink{lauriv@guzik.com}
	\end{itemize}
}

\end{document}