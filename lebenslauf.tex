\documentclass[multipage]{my_resume}
%\documentclass{my_resume}

\usepackage{tabularx}

\begin{document}

\pagenumbering{gobble}% Remove page numbers (and reset to 1)

\HeaderBar{%
	Jonathan Spitz, Phd.
}{%
	%Saint-Georges Stra{\ss}e 82\\54000 Nancy -- Frankreich
	%
	Einen kreativer Roboter Entwickler, mit einer Leidenschaft f{\"u}r die L{\"o}sung komplexer Probleme. Erfahrener Forscher, der hilft humanoiden Roboter zu lernen wie man sich bewegt.
	%
	%A creative roboticist, with a passion for solving complex problems.\\%
	%Skilled researcher helping humanoid robots learn how to move.
}{%
	http://jonathanspitz.weebly.com/
}{%
	https://www.linkedin.com/in/spitzjonathan/
}{%
	+33-624-190544
}{%
	spitz.jonathan@gmail.com
}

\MainContent{white}{
	\section*{Pers{\"o}nliche Daten}%
	\ifmultipage
	\textbf{Geburtsdatum:} 07.06.1985\\%
	\textbf{Geburtsort:} Mendoza, Argentinien\\%
	\textbf{Staatsangeh{\"o}rigkeit:} Argentinisch - Israelisch\\%
	\textbf{Femilienstand:} Ledig\\%
	\else
		\begin{tabularx}{1.0\textwidth}{@{}lXr@{}}
			\textbf{Geburtsdatum:} & & 07.06.1985\\%
			\textbf{Geburtsort:} & & Mendoza, Argentinien\\%
			\textbf{Staatsangeh{\"o}rigkeit:} & & Argentinisch - Israelisch\\%
			\textbf{Femilienstand:} & & Ledig\\%
		\end{tabularx}
	\fi
	\section*{Berufliche Erfahrungen}
	\PositionEntry{%
		Februar 2017 -- Laufend%
	}{%
		Postdoktoranden bei Inria%
	}{%
		Nancy, Frankreich%
	}{%
		\begin{itemize}
			\item Entwickelte Ganzk{\"o}rper-Kontrollalgorithmen f{\"u}r humanoide Roboter in MATLAB/C++.%
			%Developed whole-body control algorithms for humanoid robots in MATLAB/C++.%
			\item Implementierte einen Maschinenlernen Ansatz zur Verbesserung der Leistung des Kontrollalgorithmus in wenigen Versuchen.%
			%Implemented a machine learning approach to improve the controller's performance within a few trials.%
			\item Arbeitete mit einem simulierten und echten iCub humanoiden Roboter, der {\"u}ber 50 Freiheitsgrade besitzt.%
			%Worked with a simulated and real iCub humanoid robot, possessing over 50 degrees of freedom.%
			\item Unterst{\"u}tzte die Entwicklung eines Miniatur-Explorationsroboters als Nebenprojekt.
			%Aided the development of a miniature exploration robot, as a secondary project.
		\end{itemize}
	}\EntrySpacing
	
	\PositionEntry{%
		Juni 2016 -- Dezember 2016%
	}{%
		Algorithmus Ingenieur bei Cogniteam%
	}{%
		Petach Tikva, Israel%
	}{%
		\begin{itemize}
			\item Bearbeitete auf Computer-Vision-Algorithmen f{\"u}r autonome Navigation.%
			%Worked on computer vision algorithms for autonomous navigation.%
			\item Entwickelte eine Simulationsumgebung zur Erzeugung von k{\"u}nstlichen Daten f{\"u}r die Algorithmuspr{\"u}fung.
			%Designed a simulation environment to generate artificial data for algorithm testing.%
			\item Implementierte Struktur-aus-Bewegung Algorithmen mit Stereo und Multi-Ansichten Ans{\"a}tzen.
			%Implemented structure-from-motion algorithms using stereo and multiple-view approaches.%
		\end{itemize}%
	}\EntrySpacing%

	\PositionEntry{%
		M{\"a}rz 2015 -- Oktober 2015%
	}{%
		Finalist im BizTEC Unternehmerwettbewerb%
		%Finalist at BizTEC Entrepreneurship Competition%
	}{%
		Haifa, Israel%
	}{%
		\begin{itemize}
			\item Nahm an den Sommerschul- und Beschleunigerstufen des Wettbewerbs teil. Einer von 12 Finalisten von {\"u}ber 80 Teams.
			%Participated in the summer-school and accelerator stages of the competition. One of 12 finalists out of over 80 teams.%
			\item Bearbeitete auf allen Bereichen der Lean Startup-Methode, um ein Roboter-Spielzeug auf den Markt zu bringen.
			%Worked on all areas of the lean canvas to bring a robotic toy to market.%
			\item Die Roboter erhielten Berichterstattungen an
			%Robots received coverage at
			\href{http://techcrunch.com/2015/01/14/designer-builds-cute-little-3d-printed-robots-you-know-for-kids/}{TechCrunch}, \href{http://spectrum.ieee.org/automaton/robotics/robotics-hardware/video-friday-humanoids-sing-eyeball-robot-kuka-ping-pong}{IEEE Spectrum}.
		\end{itemize}%
	}\EntrySpacing%

	\PositionEntry{%
		Oktober 2014 -- Mai 2016%
	}{%
		Rapid Prototyping von Laufrobotern
		%Rapid prototyping of walking robots%
	}{%
		Haifa, Israel%
	}{%
		\begin{itemize}
			\item Entwickelte 3D-gedruckte Lauf- und R{\"a}derrobotern mit mechanischen und elektronischen Bauteilen.
			%Developed 3D-printed walking and wheeled robots including mechanical and electronic components.%
			\item Erstellte einen nachgiebigen Biped Robot, der auf Serien-elastischen Stellantrieben basiert. 
			%Designed a compliant biped robot based on series-elastic actuators.%
			\item Programmierte Kontrollalgorithmen und Bluetooth-Kommunikation in Arduino.
			%Programmed control loops and Bluetooth communication on Arduino.%
%			\item Robots received coverage at \href{http://techcrunch.com/2015/01/14/designer-builds-cute-little-3d-printed-robots-you-know-for-kids/}{TechCrunch}, \href{http://spectrum.ieee.org/automaton/robotics/robotics-hardware/video-friday-humanoids-sing-eyeball-robot-kuka-ping-pong}{IEEE Spectrum}.
		\end{itemize}%
	}\EntrySpacing%

	\PositionEntry{%
		August 2012 -- Juni 2013%
	}{%
		DARPA Virtual Robotics Challenge%
	}{%
		Haifa, Israel%
	}{%
		\begin{itemize}
			\item Erstellte Kontrollalgorithmen f{\"u}r die Fortbewegung des ATLAS humanoiden Roboters.%
			%Created controllers for locomotion of the ATLAS humanoid robot.
			\item Gearbeitete in Ubuntu Linux mit ROS-Gazebo und C++/python.%
			%Worked in Ubuntu Linux using ROS-Gazebo and C++/python.
			\item Erledigte die Mobilit{\"a}tsaufgabe (Schlammgrube, H{\"u}gel und Schuttgebiete) mit robusten kriechenden Gangarten.%
			%Completed the dismounted mobility task (mud-pit, hills and debris areas) using robust crawling gaits.
			\item Entwickelte einen robusten, dynamischen Ansatz, um das Auto mit ATLAS zu betreten.%
			%Created a robust, dynamic approach to enter the car using ATLAS.
		\end{itemize}
	}\EntrySpacing

	\PositionEntry{%
		April 2009 -- Dezember 2009%
	}{%
		Bachelor-Projekt: Robotics Simulation Environment (ROSIE)%
	}{%
		Haifa, Israel%
	}{%
		\begin{itemize}
			\item Erstellte eine Robotik-Simulationsumgebung f{\"u}r den Robotics 101 Kurs, codiert in C++ und OpenGL.%
			%Created a robotics simulation environment for the Robotics 101 course, coded in C++ and OpenGL.
			\item Eigenschaften enthalten Vor-und R{\"u}ckw{\"a}rts-Kinematik, Pfad Planung und Jacobian Berechnung.%
			%Features included forward and backward kinematics, path planning and Jacobian calculation. 
		\end{itemize}
	}\EntrySpacing

	\PositionEntry{%
		August 2007 -- September 2007%
	}{%
		Praktikum bei Guzik Technical Enterprises%
	}{%
		Mountain View, CA, USA%
	}{%
		\begin{itemize}
			\item Entwarf SolidWorks Modelle f{\"u}r ein neues Produkt, einschlie{\ss}lich eines Racks, eines Motherboards und mehrerer angeschlossener Boards.%
			%Designed SolidWorks models for a new product, including a rack, a motherboard and several connected	boards.
			\item Gearbeitete mit der Elektrotechnik-Abteilung zur Auswahl und Positionierung von Steckverbindern f{\"u}r alle Projektboards.%
			%Worked with the electrical engineering department to select and position connectors for all the project's boards.
		\end{itemize}
	}\EntrySpacing

	\section*{Schulbildung}
	\DiplomaEntry{%
		Oktober 2009 -- Oktober 2015
	}{
		Phd. in Maschinenbau%
	}{%
		\textbf{,} Direkte Phd. Track f{\"u}r hervorragende Studenten, Notendurchschnitt: 94/100.%
	}{
		Technion, Fakult{\"a}t f{\"u}r Maschinenbau%
	}{
		Haifa, Israel%
	}{
		Forschungstitel: Bio-Inspired Controllers for Dynamic Locomotion\\%
		Berater: ao. Prof. Miriam Zacksenhouse%
		\begin{itemize}
			\item Entwickelte biologisch inspirierte Kontrollalgorithmen die stabile Laufg{\"a}nge f{\"u}r einfache und komplexe Biped-Modelle erzeugen.%
			%Developed biologically inspired controllers that generate stable walking gaits for simple and complex biped models.%
			\item Verbesserte Robustheit zehnfach mit minimalem Umgebung-Feedback.%
			%Improved robustness ten-fold using minimal environmental feedback.%
			\item Entwickelte eine Reihe von verschiedenen G{\"a}ngen mit multi-objektiven genetischen Algorithmen.
			%Evolved a range of different gaits using multi-objective genetic algorithms.%
		\end{itemize}
	}\EntrySpacing
	
	\SimpleDiplomaEntry{%
		M{\"a}rz 2005 -- Februar 2010%
	}{
		BSc in Maschinenbau%
	}{%
		Technion, Fakult{\"a}t f{\"u}r Maschinenbau%
	}{
		Haifa, Israel%
	}
}{
	\section*{F{\"a}higkeiten und Zusatzqualifikationen}
	\SkillEntry{Software}{MATLAB (+10 Jahre), Arduino (+5 Jahre), C++ (+5 Jahre), Python (4 Jahre), ROS (1 Jahr).}
	\SkillEntry{Hardware}{SolidWorks (+10 Jahre), 3D Drucken (3 Jahre), Rapid-Prototyping (3 Jahre).}
	\SkillEntry{Sprachen}{Spanisch (C2), Englisch (C2), Hebr{\"a}isch (C1), Deutsch (B1), Franz{\"o}sisch (B1).}%
	\vspace{-8pt}
	
	\section*{Interesse}
	Startups, Aktienmarkt, Wandern, Sci-Fi.
	
	\section*{Stipendien und Auszeichnungen}
	\underline{Juni 2014}: Verstorbene Prof. Roland Weill Auszeichnung f{\"u}r herausragende Leistungen in der Doktorandenforschung.\\%
	\underline{Dezember 2012}: Fine's Stipendium f{\"u}r das 1. Jahr Phd. Studenten.\vspace{0.15em}\\%
	\underline{M{\"a}rz 2007}: Auszeichnung des Dekans f{\"u}r herausragende Notendurchschnitt (4. Semester).
	
	\section*{Patente}
	\BibEntry{1}{``Robot, device and a method for a central pattern generator (CPG) based control of a movement of the robot'', United States Patent Application 20140031986, Januar 2014.}
	
	\section*{Liste der Publikationen}
	\underline{\smash{Peer-Review Artikeln in Fachzeitschrift}}\vspace{0.15em}\\%
	\BibEntry{1}{J. Spitz, R. Yakar, M. Zacksenhouse,``Machine Learning Tools Facilitate Parameter Tuning of Networks of Matsuoka Neurons'', in preparation, August 2017.}\\%
	\BibEntry{2}{J. Spitz, A. Evstrachin, M. Zacksenhouse,``Minimal Feedback to a Rhythm Generator Improves the Robustness to Slope Variations of a Compass Biped'', Bioinspiration \& Biomimetics, August 2015.}\\%
	\BibEntry{3}{J. Spitz, E. Sidorov, M. Zacksenhouse, ``Humanoids Can Take Advantage of Crab-Walking Gaits'', International Journal of Humanoid Robotics, Dezember 2014.}
	\EntrySpacing

	\noindent
	\underline{\smash{Peer-Review Artikeln in Tagungsunterlagen}}\vspace{0.15em}\\%
	\BibEntry{1}{J. Spitz, K. Bouyarmane, S. Ivaldi, J.B. Mouret, ``Trial-and-Error Learning of Repulsors for Humanoid QP-based Whole-Body Control'', eingereicht zu IEEE-RAS
		Humanoids. 2017.}\\%
	\BibEntry{2}{J. Spitz, Y. Or, M. Zacksenhouse, ``Towards a Biologically Inspired Open Loop Controller for Dynamic Biped Locomotion'', In Proceedings of the 2011 International Conference on Robotics and Biomimetics.}

	\section*{Liste der Konferenzen}
	\underline{\smash{Vortr{\"a}ge}}\vspace{0.15em}\\%
	\BibEntry{1}{``Bio-Inspired Controllers for Dynamic Walking: CPG Enhanced with Minimal Feedback'', Israeli Conference on Robotics, 2013.}\\%
	\BibEntry{2}{``A Biologically Inspired Controller for Dynamic Bipedal Locomotion'', Graduate Students in Control Conference, Israel, 2012.}\\%
	\BibEntry{3}{``Towards a Biologically Inspired Open Loop Controller for Dynamic Biped Locomotion'', IEEE International Conference on Robotics and Biomimetics, Phuket, Thailand, 2011.}
	\EntrySpacing
	
	\noindent
	\underline{\smash{Posterpr{\"a}sentationen}}\vspace{0.15em}\\%
	\BibEntry{1}{``Analytical and numerical tools for limit-cycle evaluation in hybrid dynamical system'', Dynamic Walking Conference, ETH Z\"urich, der Schweiz, 2014.}\\%
	\BibEntry{2}{``Bio-Inspired Controllers for Dynamic Locomotion'', Dynamic Walking Conference, CMU Pittsburgh, USA, 2013.}\\%
	\BibEntry{3}{``A Biologically Inspired Controller For Dynamic Biped Locomotion'', The Eighth Computational Motor Control Workshop at Ben-Gurion University of the Negev, Israel, 2012.}
	
	\section*{Kontaktinformationen f{\"u}r Referenzen}
	\begin{itemize}
		\item Jean-Baptiste Mouret, Phd. (Forschungsdirektor bei Inria), \EmailLink{jean-baptiste.mouret@inria.fr}
		\item Serena Ivaldi, Phd. (Wissenschaftler bei Inria), \EmailLink{serena.ivaldi@inria.fr}
		\item Miriam Zacksenhouse, Phd. (ao. Prof. bei Technion), +972-54-5820404, \EmailLink{mermz@technion.ac.il}
%		\item Yizhar Or, Phd. (ao. Prof. bei Technion), +972-77-8875493, 	
%		\EmailLink{izi@technion.ac.il}
		\item Ari Yakir (Projektleiter bei Cogniteam), \EmailLink{ari@cogniteam.com}
		\item Lauri Viitas (VP f{\"u}r Produkt- und Gesch{\"a}ftsentwicklung bei Guzik), +1-650-625-8000, \EmailLink{lauriv@guzik.com}
	\end{itemize}
}

\end{document}